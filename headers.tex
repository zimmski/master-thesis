\documentclass[11pt,a4paper,titlepage,
chapterprefix,headsepline,parskip,pdftex,
,pointlessnumbers,bibtotoc]{scrbook}

%%% Absätze bei tieferen Ebenen einschalten
\makeatletter %% Sonderbedeutung von @ aufheben
\renewcommand{\paragraph}{\@startsection
   {paragraph} % name
   {4} % ebene
   {0mm} % einzug
   {-\baselineskip} % vorabstand
   {0.1\baselineskip} % nachabstand
   {\normalfont\normalsize\bfseries}} % stil
\makeatother %% Sonderbedeutung von @ wieder

\makeatletter %% Sonderbedeutung von @ aufheben
\renewcommand{\subparagraph}{\@startsection
   {subparagraph} % name
   {5} % ebene
   {0mm} % einzug
   {-\baselineskip} % vorabstand
   {0.1\baselineskip} % nachabstand
   {\normalfont\normalsize\bfseries}} % stil
\makeatother %% Sonderbedeutung von @ wieder

\usepackage{setspace}
\onehalfspacing

\usepackage[pdftex]{graphicx}

%my imports:
\usepackage{subfig}
\usepackage{afterpage}
\usepackage{algorithmic}
\usepackage[para,online,flushleft]{threeparttable} % Custom captions under/above floats in tables or figures.
\usepackage[chapter]{algorithm}
\usepackage{mathtools}
\usepackage{float}
\usepackage{listings}
\usepackage{graphicx}
\usepackage{boxedminipage}
\usepackage{framed}
\usepackage{amssymb}
\usepackage{amsmath} % Includes \nobreakdash and other stuff.
\usepackage{tabulary}
\usepackage[pdftex]{colortbl}
%my theorems:
%\newtheorem{definition}{Definition}
\newtheorem{definition}{Definition}
\numberwithin{definition}{chapter}

% for colours
\usepackage[pdftex]{color}

\usepackage[colorlinks=true,
    linkcolor=blue,
    citecolor=blue,
    pagecolor=blue,
    urlcolor=blue,
    breaklinks=true,
    bookmarksnumbered=true,
    hypertexnames=false,
    pdfpagemode=UseOutlines,
    pdfview=FitH,
    plainpages=false,
    pdfpagelabels,
    bookmarks=true,
    linktocpage=true]{hyperref}


% To-Do Befehl
\newcommand{\todo}[1]{\textcolor{red}{\textbf{To-Do:} #1}}

% Interner-Link-Befehl
\newcommand{\internerLink}[1]{\hyperref[#1]
{Siehe \ref*{#1}~\nameref{#1} auf S.~\pageref{#1}}}

% Interner-Link-Befehl 2
\newcommand{\ffinternerLink}[1]{\hyperref[#1]
{Siehe S.~\pageref{#1}ff}}

% Interner-Link-Befehl x
\newcommand{\xinternerLink}[1]{\hyperref[#1]
{\ref*{#1}~\nameref{#1} auf S.~\pageref{#1}}}


%%% continous footnote
\newcounter{cfootnotecounter}
\newcommand{\cfootnote}[1]{\stepcounter{cfootnotecounter}
\footnote[\value{cfootnotecounter}]{#1}}

%%% Bild Befehle
\newcommand{\bild}[3]{\begin{figure}[htb] \begin{center}
\includegraphics{#1}
\caption{#2} \label{#3} \end{center} \end{figure}}

\newcommand{\bildTabelle}[3]{\begin{table}[htb] \begin{center}
\includegraphics{#1}
\caption{#2} \label{#3} \end{center} \end{table}}

\newcommand{\bildE}[5]{\begin{figure}[hb] \begin{center}
\includegraphics[height=#2, angle=#3]{#1}
\caption{#4} \label{#5} \end{center} \end{figure}}

\newcommand{\tabelle}[3]{\begin{table}[htb] \begin{center}
\input{#1}
\caption{#2} \label{#3} \end{center} \end{table}}

\flushbottom

% change page settings
\setlength{\hoffset}{0mm} \setlength{\voffset}{0mm}
% \setlength{\evensidemargin}{14.6mm}
% \setlength{\oddsidemargin}{14.6mm}
\setlength{\topmargin}{-20mm}
\setlength{\headheight}{15mm} \setlength{\headsep}{9mm}
\setlength{\textheight}{242mm} \setlength{\textwidth}{145mm}
\setlength{\footskip}{10mm}
%%% Nachfolgendes nicht notwendig wg. Klassenoption parskip
%\setlength{\parskip}{3ex plus0.5ex minus0.5ex}
%\setlength{\parindent}{0mm}

%%% Abstände von float-Umgebungen
% \setlength{\textfloatsep}{25pt plus5pt minus5pt}
% \setlength{\intextsep}{25pt plus5pt minus5pt}

%%% Gliederungs-Nummern in den Rand schreiben
\renewcommand*{\othersectionlevelsformat}[1]{%
\llap{\csname the#1\endcsname\autodot\enskip}}

%%% In Kopfzeile nur Kapitel-Text ohne "Kapitel x"
\renewcommand*{\chaptermarkformat}{}

%%% Formatierung von chapter ändern
\setkomafont{chapter}{\Huge}
\RedeclareSectionCommand[beforeskip=50pt]{chapter}
\renewcommand*{\chapterformat}{\LARGE{\chapappifchapterprefix{\ }\thechapter\autodot\enskip}}

%%% Kopfzeile
\usepackage[automark]{scrpage2}

\clearscrheadings \clearscrplain \clearscrheadfoot
\pagestyle{scrheadings}
\ohead{\pagemark}
\ihead{\headmark}
\cfoot{}

%%% Formatierung von Kapitel-Seiten
\renewcommand*{\chapterpagestyle}{scrheadings}

%% Gliederung TOC und Nummerierungstiefe
\setcounter{tocdepth}{\subsubsectionlevel}
\setcounter{secnumdepth}{\subsubsectionlevel}

%%% Schriftarten
\addtokomafont{chapter}{\sffamily}
\addtokomafont{sectioning}{\rmfamily}

% Sprache
\usepackage[german,english]{babel}
% Verwenden von T1 Fonts
\usepackage[T1]{fontenc}
% Eingabe von Umlauten
\usepackage[utf8x]{inputenc}
\usepackage{ae}

% URLs
\usepackage{url}

%%% Schusterjungen und Hurenkinder
\clubpenalty = 10000
\widowpenalty = 10000 \displaywidowpenalty = 10000


%%% Einbinden von kompletten PDF-Seiten
\usepackage{pdfpages}

% Quotes
\usepackage{csquotes}

% Source code highlighting
\usepackage{xcolor}
\usepackage{minted} % The background color is NEEDED because otherwise there is no margin-top for the frame
\renewcommand\theFancyVerbLine{\small\arabic{FancyVerbLine}}
\newminted{bash}{
  linenos,
  numbersep=5pt,
  breaklines,
  bgcolor=white,
}
\newminted{diff}{
  linenos,
  numbersep=5pt,
  breaklines,
  bgcolor=white,
}
\newminted{go}{
  linenos,
  numbersep=5pt,
  breaklines,
  tabsize=4,
  bgcolor=white,
}
\newminted{text}{
  linenos,
  numbersep=5pt,
  breaklines,
  bgcolor=white,
}
\newminted{xml}{
  linenos,
  numbersep=5pt,
  breaklines,
  bgcolor=white,
}
\AtBeginEnvironment{minted}{\renewcommand{\fcolorbox}[4][]{#4}} % Ignore syntax errors

% Smiley, because I cannot enter unicode code points into Latex...
\usepackage{wasysym}

% For left, right and center attributes of includegraphics
\usepackage[export]{adjustbox}

% Landscape pages
\usepackage{lscape}

% Headers, captions, ... with title case.
\usepackage{titlecaps}

% \let\nottitlecapsection=\chapter % The chapter command does not work with this trick and I have no time to fix it.
% \def\chapter#1{\nottitlecapchapter{\titlecap{#1}}}
\let\nottitlecapsection=\section
\def\section#1{\nottitlecapsection{\titlecap{#1}}}
\let\nottitlecapsubsection=\subsection
\def\subsection#1{\nottitlecapsubsection{\titlecap{#1}}}
\let\nottitlecapcaption=\caption
\def\caption#1{\nottitlecapcaption{\titlecap{#1}}}

\Addlcwords{a an and at but by for from in nor of on or the to}

% Center every figure and table.
\makeatletter
\g@addto@macro\@floatboxreset\centering
\makeatother

\tolerance=2000                 % Zur Vermeidung von "overfull hbox"
\emergencystretch 20pt          % " besser als sloppy
