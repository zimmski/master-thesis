\chapter{\textsc{Tavor Format}}
\label{chapter:tavorFormat}

The \textsc{Tavor format} is an EBNF-like notation which allows the definition of data, such as file formats and protocols, without the need of programming. It is the default format of the Tavor CLI of Chapter~\ref{chapter:tavorCLI} and supports every feature of the framework, which has been introduced in Chapter~\ref{chapter:tavorFramework}. Each of the demonstrated features of the \textsc{Tavor format} is required to fully define the AIGER format.

The format is Unicode text encoded in UTF-8 and consists of terminal and non-terminal symbols which are called \texttt{tokens} throughout the \textsc{Tavor framework}. A more general definition of \texttt{tokens} can be found in Section~\ref{sec:tokens}.

\section{Token Definition}
\label{sec:tokenDefinition}

Every \texttt{token} in the format belongs to a definition of \texttt{non-terminal token} which consists of a unique case-sensitive name and its definition part. Both are separated by exactly one equal sign. Syntactical white spaces are ignored. Every \texttt{token definition} is by default declared in one line. A line ends with a new-line character.

To give an example, the format in Listing~\ref{lst:tavorHelloWorld} declares the \texttt{START token} with the \texttt{String Token} \enquote{Hello World} as its definition.

\begin{listing}
\caption{\textsc{Tavor}'s Hello World}
\label{lst:tavorHelloWorld}
\begin{gocode}
START = "Hello World"
\end{gocode}
\end{listing}

\texttt{Token names} have the following constraints: Each \texttt{token name} has to start with a letter, and can only consist of letters, digits and the underscore sign \enquote{\char`_}. Additionally, a \texttt{token name} has to be unique. It is also not allowed to declare a \texttt{token} and never use it. The \texttt{START token} is the only exception, which is used as the entry point of the format. Hence it defines the beginning of the format and is therefore required for every format definition.

Sequential listed \texttt{tokens} in the definition part of a \texttt{token definition} are automatically concatenated. The example in Listing~\ref{lst:concatenationExample} concatenates to the string \enquote{This is a String Token and this 123 was a number token.}.

\begin{listing}
\caption{Example for a token concatenation}
\label{lst:concatenationExample}
\begin{gocode}
START = "This is a String Token and this " 123 " was a number token."
\end{gocode}
\end{listing}

A \texttt{token definition} can be sometimes too long or poorly readable, it can be therefore split into multiple lines by using a comma before the newline character as shown in Listing~\ref{lst:multiLineTokenDefinitionExample}. The \texttt{token definition} ends at the string \enquote{definition.} since there is no comma before the newline character. Listing~\ref{lst:multiLineTokenDefinitionExample} also highlights that syntactical white spaces are ignored and can be used to make a format definition more human-readable.

\begin{listing}
\caption{Example for a multi-line token definition}
\label{lst:multiLineTokenDefinitionExample}
\begin{gocode}
START = "This",
        "is",
        "a",
        "multi line",
        "definition."
\end{gocode}
\end{listing}

The \textsc{Tavor format} supports two kinds of comments. \texttt{Line comments} start with the character sequence \enquote{//} and end at the next new-line character. \texttt{General comments} start with the character sequence \enquote{/*} and end at the character sequence \enquote{*/}. A general comment can therefore contain new-line characters and can be used between \texttt{token definition} and \texttt{tokens}. Both kinds of comments are illustrated in Listing~\ref{lst:commentsExample}.

\begin{listing}
\caption{Example for different kinds of comments}
\label{lst:commentsExample}
\begin{gocode}
/*

This is a general comment
which can have
multiple lines.

*/

START = "This is a string." // This is a line comment.

// This is also a line comment.
\end{gocode}
\end{listing}

\section{Terminal Tokens}
\label{sec:terminalTokens}

\texttt{Terminal tokens} are the constants of the \textsc{Tavor format}. The format supports two kinds of \texttt{terminal tokens}: numbers and strings. Every other token that is not a \texttt{terminal token}, such as \texttt{tokens} of definitions, is called a \texttt{non-terminal token}.

\texttt{Number tokens} allow only positive decimal integers, which are written as a sequence of digits as shown in Listing~\ref{lst:numberTokenExample}.

\begin{listing}
\caption{Example for a number token}
\label{lst:numberTokenExample}
\begin{gocode}
START = 123
\end{gocode}
\end{listing}

\texttt{String Tokens} are character sequences between double quote characters and can consist of any UTF-8 encoded character except the new-line, the double quote and the backslash characters which have to be escaped with a backslash character. An example can be seen in Listing~\ref{lst:stringTokenExample}.

\begin{listing}
\caption{Example for a String Token}
\label{lst:stringTokenExample}
\begin{gocode}
START = "The next word is \"quoted\" and next is a new line\n"
\end{gocode}
\end{listing}

Since \textsc{Tavor} is using \textsc{Go}'s text parser as foundation of its format parsing, the same rules for \enquote{interpreted string literals} as defined in \textsc{Go}'s language specification~\cite{2017_go_spec} apply to \texttt{String Tokens}.

Empty \texttt{String Token} are forbidden and lead to a format parsing error. The reason for this exceptions is the way \texttt{tokens} are utilized during parsing and delta-debugging, and is described in the repeat groups of Paragraph~\ref{par:groupRepeat} in more detail.

\section{Embedding of Tokens}
\label{sec:tokenEmbedding}

\texttt{Non-terminal tokens} can be embedded in the definition part by using the name of the referenced \texttt{token}. The example in Listing~\ref{lst:embeddingTokens} embeds the \texttt{token} \enquote{Text} into the \texttt{START token}. \texttt{Token names} declared in the global scope of a format definition can be used throughout the format regardless of their declaration position. Terminal and non-terminal tokens can be mixed as illustrated in Listing~\ref{lst:mixedTokensExample}.

\begin{listing}
\caption{Example for embedding tokens}
\label{lst:embeddingTokens}
\begin{gocode}
START = Text

Text = "Some text"
\end{gocode}
\end{listing}

\begin{listing}
\caption{Example for terminal and non-terminal tokens mixed in one format}
\label{lst:mixedTokensExample}
\begin{gocode}
Dot = "."

First  = 1 Dot
Second = 2 Dot
Third  = 3 Dot

START = First ", " Second " and " Third
\end{gocode}
\end{listing}

The \textsc{Tavor framework} and therefore the \textsc{Tavor format} differentiate between a \texttt{token reference}, which is the embedding of a \texttt{token} in a definition, and a \texttt{token usage}, which is the execution of a \texttt{token} during an operation such as fuzzing or delta-debugging. Listing~\ref{lst:tokenUsageTokenReferenceExample} illustrates the difference between a \texttt{token reference} and a \texttt{token usage}. The format defines two \texttt{tokens} called \enquote{Choice} and \enquote{List}. There exists one \texttt{token reference} of \enquote{Choice}, which can be found in the \enquote{List} definition, and two for \enquote{List}, which are both in the \texttt{START token} definition. Although \enquote{Choice} is in a \texttt{repeat group}, which in this example repeats the token \enquote{Choice} exactly twice, it is only referenced once. \enquote{List} has two \texttt{token usages} in this format while \enquote{Choice} has 4. Every \enquote{List} \texttt{token} does have two \enquote{Choice} usages because of the \texttt{repeat group} in the definition of \enquote{List}.

\begin{listing}
\caption{Example for a token reference and a token usage}
\label{lst:tokenUsageTokenReferenceExample}
\begin{gocode}
Choice = "a" | "b" | "c"

List = +2(Choice)

START = "1. list: " List "\n",
        "2. list: " List "\n"
\end{gocode}
\end{listing}

\section{Alternations}
\label{sec:alternation}

\texttt{Alternations} are defined by the pipe character \enquote{|} which separates two \texttt{alternation terms}. The example in Listing~\ref{lst:alternationExample} defines that the token \texttt{START} can either hold the strings \enquote{a}, \enquote{b} or \enquote{c}. An \texttt{alternation term} has its own scope which means that a sequence of \texttt{tokens} can be used.

\begin{listing}
\caption{Example for the alternation token}
\label{lst:alternationExample}
\begin{gocode}
START = "a" | "b" | "c"
\end{gocode}
\end{listing}

\texttt{Alternation terms} can be empty which allows more advanced definitions of formats. Listing~\ref{lst:alternationLoop} illustrates how to use an alternation to define a loop which can either hold the empty string or the strings \enquote{a}, \enquote{b}, \enquote{ab}, \enquote{aab} or any amount of \enquote{a} characters ending with an optional \enquote{b} character.

\begin{listing}
\caption{Example for loop using an alternation}
\label{lst:alternationLoop}
\begin{gocode}
A = "a" A | B |
B = "b"

START = A
\end{gocode}
\end{listing}

\section{Groups}
\label{sec:groups}

\texttt{Tokens} can be arranged using \texttt{token groups} by using parenthesis beginning with the opening parenthesis character \enquote{(} and ending with the closing parenthesis character \enquote{)}. A \texttt{group} is a \texttt{token} on its own and can be therefore mixed with other \texttt{tokens}. Additionally, a \texttt{group} starts a new scope between its parenthesis and can therefore hold a sequence of \texttt{tokens}. The \texttt{tokens} between the parenthesis are called the \texttt{group body}. Listing~\ref{lst:groupToken} illustrates the usage of a \texttt{token group} by declaring that the \texttt{START token} can either hold the string \enquote{a c} or \enquote{d c}. \texttt{Groups} can be nested as illustrated in Listing~\ref{lst:nestedGroups}, which defines a number with one to three digits. \texttt{Groups} can have modifiers which give a \texttt{group} additional abilities. The following sections introduces these modifiers.

\begin{listing}
\caption{Example for a group token}
\label{lst:groupToken}
\begin{gocode}
START = ("a" | "d") " c"
\end{gocode}
\end{listing}

\begin{listing}
\caption{Example for nested groups}
\label{lst:nestedGroups}
\begin{gocode}
Digit = 0 | 1 | 2 | 3 | 4 | 5 | 6 | 7 | 8 | 9

START = Digit (Digit (Digit | ) | )
\end{gocode}
\end{listing}

\paragraph{Optional group}
\label{subsec:groupOptional}

The \texttt{optional group} has the question mark \enquote{?} as modifier and allows the whole \texttt{group token} to be optional. The \texttt{START token} in Listing~\ref{lst:optionalGroup} can either hold the strings "a" or "a b".

\begin{listing}
\caption{Example for an optional group}
\label{lst:optionalGroup}
\begin{gocode}
START = "a" ?(" b")
\end{gocode}
\end{listing}

\paragraph{Repeat group}
\label{par:groupRepeat}

The default modifier for the \texttt{repeat group} is the plus character \enquote{+}. The repetition is executed by default at least once. In Listing~\ref{lst:groupRepeat} the string \enquote{a} is repeated and the \texttt{START token} can therefore hold any amount of \enquote{a} characters but at least one.

\begin{listing}
\caption{Example for a repeat group}
\label{lst:groupRepeat}
\begin{gocode}
START = +("a")
\end{gocode}
\end{listing}

Since repeated empty \texttt{tokens} lead to infinite steps during parsing and delta-debugging, a parsing error is issued if such \texttt{tokens} are encountered. This includes \texttt{optional groups} or \texttt{alternations} with at least one empty term inside a \texttt{Repeat Token}. The \texttt{repeat group} repeats by default from one to infinite times. The repetition can be altered with arguments to the modifier. Listing~\ref{lst:groupFixedRepeat} repeats the string \enquote{a} exactly twice. The \texttt{START token} can therefore only hold the string \enquote{aa}. It is also possible to define a repetition range. Listing~\ref{lst:groupRangedRepeat} repeats the string \enquote{a} at least twice but at most four times. This means that the \texttt{START token} can either hold the strings \enquote{aa}, \enquote{aaa} or \enquote{aaaa}. The \enquote{from} and \enquote{to} arguments can be empty too. They are then set to their default values. Listing~\ref{lst:groupEmptyFromArgumentRepeatGroup} repeats the string \enquote{a} at least once and at most four times. Listing~\ref{lst:groupEmptyToArgumentRepeatGroup} repeats the string \enquote{a} at least twice. Listing~\ref{lst:groupOptionalRepeat} illustrates the \texttt{optional repeat group} modifier. The \texttt{START token} can either hold the strings \enquote{a}, \enquote{ab}, \enquote{abb} or any amount of \enquote{b} characters prefixed by an "a" character.

\begin{listing}
\caption{Example for a fixed repeat group}
\label{lst:groupFixedRepeat}
\begin{gocode}
START = +2("a")
\end{gocode}
\end{listing}

\begin{listing}
\caption{Example for a ranged repeat group}
\label{lst:groupRangedRepeat}
\begin{gocode}
START = +2,4("a")
\end{gocode}
\end{listing}

\begin{listing}
\caption{Example for an empty ``from'' argument for a ranged repeat group}
\label{lst:groupEmptyFromArgumentRepeatGroup}
\begin{gocode}
START = +,4("a")
\end{gocode}
\end{listing}

\begin{listing}
\caption{Example for an empty ``to'' argument for a ranged repeat group}
\label{lst:groupEmptyToArgumentRepeatGroup}
\begin{gocode}
START = +2,("a")
\end{gocode}
\end{listing}

\begin{listing}
\caption{Example for an optional repeat group}
\label{lst:groupOptionalRepeat}
\begin{gocode}
START = "a" *("b")
\end{gocode}
\end{listing}

\paragraph{Permutation group}
\label{subsec:groupPermutation}

The \enquote{@} is the permutation modifier which is combined with an alternation in the group body. Each alternation term will be executed exactly once but the order of execution is non-relevant. In Listing~\ref{lst:groupPermutation} the \texttt{START token} can either hold \enquote{123}, \enquote{132}, \enquote{213}, \enquote{231}, \enquote{312} or \enquote{321}.

\begin{listing}
\caption{Example for a permutation group}
\label{lst:groupPermutation}
\begin{gocode}
START = @(1 | 2 | 3)
\end{gocode}
\end{listing}

\section{Character Classes}
\label{sec:characterClasses}

\texttt{Character class tokens} can be directly compared to character classes of regular expressions used in most programming languages. A \texttt{character class token} begins with the left bracket \enquote{[} and ends with the right bracket \enquote{]}. The content between the brackets is called a pattern and can consists of almost any UTF-8 encoded character, escape character, special escape character and range. In general the \texttt{character class token} can be seen as a shortcut for an \texttt{alternation}. The definition in Listing~\ref{lst:characterClassExample} illustrates the usage of a \texttt{character class} by defining that the \texttt{START token} holds either the strings \enquote{a}, \enquote{b} or \enquote{c}.

\begin{listing}
\caption{Example for a character class token}
\label{lst:characterClassExample}
\begin{gocode}
START = [abc]
\end{gocode}
\end{listing}

\paragraph{Escape Characters}
\label{subsec:characterClassEscapeCharacters}

Table~\ref{table:escapeCharacters} holds all \texttt{escape characters} which are UTF-8 encoded characters that are not directly allowed within a \texttt{character class} pattern. Their equivalent escape sequence has to be used instead. Listing~\ref{lst:escapeCharacterExample} defines that the \texttt{START token} can hold only white space characters.

\begin{table}[H]
\caption{Escape characters for character classes}
\label{table:escapeCharacters}
\center
\begin{tabular}{| l | l |}
\hline
  \textbf{Character}
& \textbf{Escape sequence}
\tabularnewline
\hline
  \enquote{-}
& \enquote{\textbackslash-}
\tabularnewline
\hline
  \enquote{\textbackslash}
& \enquote{\textbackslash\textbackslash}
\tabularnewline
\hline
  form feed
& \enquote{\textbackslash{}f}
\tabularnewline
\hline
  newline
& \enquote{\textbackslash{}n}
\tabularnewline
\hline
  return
& \enquote{\textbackslash{}r}
\tabularnewline
\hline
  tab
& \enquote{\textbackslash{}t}
\tabularnewline
\hline
\end{tabular}
\end{table}

\begin{listing}
\caption{Example for escape characters in character classes}
\label{lst:escapeCharacterExample}
\begin{gocode}
START = +([ \t\n\r\f])
\end{gocode}
\end{listing}

Since some characters can be hard to type and read the \enquote{\textbackslash{}x} escape sequence can be used to define them with their hexadecimal code points. Either only two hexadecimal characters are used in the form of \enquote{\textbackslash{}x0A} or when more then two hexadecimal digits are needed the form \enquote{\textbackslash{}x\string{0AF\string}} has to be used. The second form allows up to eight digits and is therefore fully Unicode ready. Unicode usage is illustrated in Listing~\ref{lst:hexadecimalCharacterClassExample}, which either holds the Unicode character \enquote{/} or \enquote{\smiley}.

\begin{listing}
\caption{Example for hexadecimal code points in character classes}
\label{lst:hexadecimalCharacterClassExample}
\begin{gocode}
START = [\x2F\x{263A}]
\end{gocode}
\end{listing}

\paragraph{Character Class Ranges}
\label{subsec:characterClassRanges}

\texttt{Character class ranges} can be defined using the \enquote{-} character. A \texttt{range} holds all characters defined by UTF-8 starting from the character before the \enquote{-} to ending at the character after the \enquote{-}. Both characters have to be either an UTF-8 encoded or an escaped character. The starting character must have a lower value than the ending character. The usage of \texttt{Character classes} is illustrated in Listing~\ref{lst:characterClassRange} which defines a number with one digit.

\begin{listing}
\caption{Example for a character class range}
\label{lst:characterClassRange}
\begin{gocode}
START = [0-9]
\end{gocode}
\end{listing}

\paragraph{Special Escape Characters}
\label{subsec:specialEscapeCharacters}

\texttt{Special escape characters} combine many characters into one \texttt{escape character}. Table~\ref{table:specialEscapeCharacters} lists all implemented \texttt{special escape characters}.

\begin{table}[H]
\caption{Special escape characters for character classes}
\label{table:specialEscapeCharacters}
\center
\begin{tabular}{| l | l | l |}
\hline
  \textbf{Escape character}
& \textbf{Character class}
& \textbf{Description}
\tabularnewline
\hline
  \enquote{\textbackslash{}d}
& \enquote{[0-9]}
& Holds a decimal digit character
\tabularnewline
\hline
  \enquote{\textbackslash{}s}
& \enquote{[ \textbackslash{}f\textbackslash{}n\textbackslash{}r\textbackslash{}t]}
& Holds a white space character
\tabularnewline
\hline
  \enquote{\textbackslash{}w}
& \enquote{$[a-zA-Z0-9\textunderscore]$}
& Holds a word character
\tabularnewline
\hline
\end{tabular}
\end{table}

\section{Token Attributes}
\label{sec:tokenAttributes}

Some \texttt{tokens} define attributes which can be used in a definition by prefixing the dollar sign \enquote{\$} to their name and appending a dot followed by the attribute name. Listing~\ref{lst:tokenAttributeExample} illustrates \texttt{token attributes} using the \enquote{Count} attribute, which holds the count of the token's direct children. The format holds for example the string "The number 56 has 2 digits".

\begin{listing}
\caption{Example for a token attribute using the ``Count'' attribute}
\label{lst:tokenAttributeExample}
\begin{minted}[
  mathescape,
  linenos,
  numbersep=5pt,
  breaklines,
  bgcolor=white,
]{go}
Number = +([0-9])
START = "The number " Number " has " $Number.Count " digits"
\end{minted}
\end{listing}

Some \texttt{token attributes} can have arguments. A \texttt{token argument list} begins with the opening parenthesis \enquote{(} and ends with the closing parenthesis \enquote{)}, and holds \enquote{token argument parameters} which are separated by commas. All \texttt{list tokens} have for example the \enquote{Item} attribute, which holds one child of the token. The \enquote{Item} attribute has one argument which is the index of the child starting from the index zero, and is illustrated in Listing~\ref{lst:tokenAttributeParameterExample}. The format holds the string "The letter with the index 1 is b".

\begin{listing}
\caption{Example for attribute parameters using the ``Item'' attribute}
\label{lst:tokenAttributeParameterExample}
\begin{minted}[
  mathescape,
  linenos,
  numbersep=5pt,
  breaklines,
  bgcolor=white,
]{go}
Letters = "a" "b" "c"
START = "The letter with the index 1 is " $Letters.Item(1)
\end{minted}
\end{listing}

A \texttt{list token} is a \texttt{token} which has in its definition either only a sequence of \texttt{tokens} or exactly one \texttt{repeat group token}. Table~\ref{table:listTokenAttributes} shows the \texttt{token attributes} that can be utilized by \texttt{list tokens}.

\begin{table}[H]
\caption{Token attributes for list tokens}
\label{table:listTokenAttributes}
\center
\begin{tabulary}{\textwidth}{| l | l | J |}
\hline
  \textbf{Attribute}
& \textbf{Arguments}
& \textbf{Description}
\tabularnewline
\hline
  \enquote{Count}
& -
& Holds the count of the token's direct children.
\tabularnewline
\hline
  \enquote{Item}
& \enquote{i}
& Holds a direct child of the token with the index \enquote{i}.
\tabularnewline
\hline
  \enquote{Unique}
& -
& Chooses at random a direct child of the token and embeds it. The choice is unique for every reference of the token.
\tabularnewline
\hline
\end{tabulary}
\end{table}

\paragraph{Scope of Attributes}
\label{subsec:scopeOfAttributes}

The \textsc{Tavor format} allows the usage of \texttt{token attributes} as long as the referenced \texttt{token} exists in the current scope.

Two kinds of scopes exist: The \texttt{global scope} is the scope of the whole format definition. An entry of the \texttt{global scope} is set by the nearest child \texttt{token reference} to the \texttt{START token}. The \texttt{local scope} is the scope held by a definition, \texttt{group} or any other \texttt{token} which holds its own scope. \texttt{Local scopes} are initialized with entries from their parent scope at the time of the creation of the new \texttt{local scope}. Listing~\ref{lst:scopeExample} illustrates this behavior which can for example hold the string \enquote{1 a 1(1 aa 2)1 aaa 3}. The \texttt{List} token is used trice in this example. The first usage leads to the value \enquote{a}, the second to \enquote{aa} and the third to \enquote{aaa}. Depending on which list is visible in the current scope \enquote{\$List.Count} results in another value. This example showcases that the scope of token \texttt{Inner} is not visible in the scope of the \texttt{START token}.

\begin{listing}
\caption{Example for scopes}
\label{lst:scopeExample}
\begin{minted}[
  mathescape,
  linenos,
  numbersep=5pt,
  breaklines,
  bgcolor=white,
]{go}
List = +,10("a")

Inner = "(" $List.Count " " List " " $List.Count ")"

START = $List.Count " " List " " $List.Count,
        Inner,
        $List.Count " " List " " $List.Count
\end{minted}
\end{listing}

\section{Typed Tokens}
\label{sec:typedTokens}

\texttt{Typed tokens} are a functional addition to regular \texttt{token definitions} of the \textsc{Tavor format}. They provide specific functionality which can be utilized by embedding them like regular \texttt{tokens} or through their additional \texttt{token attributes}. \texttt{Typed tokens} can be defined by prefixing the dollar sign \enquote{\$} to their name. They do not have a format definition on their right-hand side. Instead, a type and optional arguments written as key-value pairs, which are separated by a colon, define the \texttt{token}. Listing~\ref{lst:typedTokenExample} illustrates \texttt{typed tokens} with the definition of an \texttt{integer token}. The format definitions holds additions with random integers as numbers such as \enquote{47245 + 6160 + 6137}.

\begin{listing}
\caption{Example typed token using an integer token}
\label{lst:typedTokenExample}
\begin{minted}[
  mathescape,
  linenos,
  numbersep=5pt,
  breaklines,
  bgcolor=white,
]{go}
$Number Int

Addition = Number " + " (Number | Addition)

START = Addition
\end{minted}
\end{listing}

The range of the \enquote{Int} type can be bounded using arguments for the definition as shown in Listing~\ref{lst:typedTokenAttributesExample}, where the \texttt{token} \enquote{Number} is an integer within the range one to ten.

\begin{listing}
\caption{Example for typed token attributes}
\label{lst:typedTokenAttributesExample}
\begin{minted}[
  mathescape,
  linenos,
  numbersep=5pt,
  breaklines,
  bgcolor=white,
]{go}
$Number Int = from: 1,
              to:   10

Addition = Number " + " (Number | Addition)

START = Addition
\end{minted}
\end{listing}

\paragraph{Typed Token \texttt{Int}}
\label{subsec:typedTokensInt}

The \texttt{Int} type implements a random integer. Its optional \texttt{token arguments} are listed in Table~\ref{table:typedTokensIntOptionalArguments} and its \texttt{token attributes} are listed in Table~\ref{table:typedTokensIntArguments}.

\begin{table}[H]
\caption{Optional Token arguments for the ``Int'' typed token}
\label{table:typedTokensIntOptionalArguments}
\center
\begin{tabular}{| l | l |}
\hline
  \textbf{Arguments}
& \textbf{Description}
\tabularnewline
\hline
  \enquote{from}
& First integer value (defaults to 0)
\tabularnewline
\hline
  \enquote{to}
& Last integer value (defaults to $2^{31} - 1$)
\tabularnewline
\hline
\end{tabular}
\end{table}

\begin{table}[H]
\caption{Token attributes for the ``Int'' typed token}
\label{table:typedTokensIntArguments}
\center
\begin{tabular}{| l | l | l |}
\hline
  \textbf{Attribute}
& \textbf{Arguments}
& \textbf{Description}
\tabularnewline
\hline
  \enquote{Value}
& -
& Embeds a new token based on its parent
\tabularnewline
\hline
\end{tabular}
\end{table}

\paragraph{Typed Token \texttt{Sequence}}
\label{subsec:typedTokensSequence}

The \texttt{Sequence} type implements a generator for integers. Its optional \texttt{token arguments} are listed in Table~\ref{table:typedTokensSequenceOptionalArguments} and its \texttt{token attributes} are listed in Table~\ref{table:typedTokensSequenceArguments}.

\begin{table}[H]
\caption{Optional Token arguments for the ``Sequence'' typed token}
\label{table:typedTokensSequenceOptionalArguments}
\center
\begin{tabular}{| l | l |}
\hline
  \textbf{Arguments}
& \textbf{Description}
\tabularnewline
\hline
  \enquote{start}
& First sequence value (defaults to 1)
\tabularnewline
\hline
  \enquote{step}
& Increment of the sequence (defaults to 1)
\tabularnewline
\hline
\end{tabular}
\end{table}

\begin{table}[H]
\caption{Token attributes for the ``Sequence'' typed token}
\label{table:typedTokensSequenceArguments}
\center
\begin{tabulary}{\textwidth}{| l | l | J |}
\hline
  \textbf{Attribute}
& \textbf{Arguments}
& \textbf{Description}
\tabularnewline
\hline
  \enquote{Existing}
& -
& Embeds a new token holding one existing value of the sequence
\tabularnewline
\hline
  \enquote{Next}
& -
& Embeds a new token holding the next value of the sequence
\tabularnewline
\hline
  \enquote{Reset}
& -
& Embeds a new token which on execution resets the sequence
\tabularnewline
\hline
\end{tabulary}
\end{table}

\afterpage{\clearpage}

\section{Expressions}
\label{sec:expressions}

\texttt{Expressions} can be used in \texttt{token definitions} and allow dynamic and complex operations using \texttt{operators} who can have different numbers of \texttt{operands}. An \texttt{expressions} starts with the dollar sign \enquote{\$} and the opening curly brace \enquote{\{} and ends with the closing curly brace \enquote{\}}. Listing~\ref{lst:expressionExample} illustrates the simplest expression which is an addition.

\begin{listing}
\caption{Example expression}
\label{lst:expressionExample}
\begin{minted}[
  mathescape,
  linenos,
  numbersep=5pt,
  breaklines,
  bgcolor=white,
]{go}
START = ${1 + 2}
\end{minted}
\end{listing}

Every \texttt{operand} of an \texttt{operator} can be a token. The usual dollar sign for a \texttt{token attribute} can be omitted inside an \texttt{expression} as illustrated in Listing~\ref{lst:tokenAttributeInExpression}.

\begin{listing}
\caption{Example for a token attribute inside an expression}
\label{lst:tokenAttributeInExpression}
\begin{minted}[
  mathescape,
  linenos,
  numbersep=5pt,
  breaklines,
  bgcolor=white,
]{go}
$Number Int

START = ${Number.Value + 1}
\end{minted}
\end{listing}


\paragraph{Arithmetic Operators}
\label{subsec:arithmeticOperators}

\texttt{Arithmetic operators} which are shown in Table~\ref{table:arithmeticOperators} have two \texttt{operands} between the \texttt{operator} sign. \texttt{Operators} always embed the right side \texttt{operand} which means that that \enquote{2 * 3 + 4} will result in \enquote{2 * (3 + 4)} and not \enquote{(2 * 3) + 4}.

\begin{table}[H]
\caption{Arithmetic operators}
\label{table:arithmeticOperators}
\center
\begin{tabular}{| l | l |}
\hline
  \textbf{Operator}
& \textbf{Description}
\tabularnewline
\hline
  \enquote{+}
& Addition
\tabularnewline
\hline
  \enquote{-}
& Subtraction
\tabularnewline
\hline
  \enquote{*}
& Multiplication
\tabularnewline
\hline
  \enquote{/}
& Division
\tabularnewline
\hline
\end{tabular}
\end{table}

\paragraph{Include Operator}
\label{subsec:includeOperator}

The \texttt{include operator} parses an external \textsc{Tavor format} file and includes its \texttt{START token}. It takes a string as its only \texttt{operand} which defines the file path of the to be included \textsc{Tavor format} file. The file path can be absolute or relative. Listing~\ref{lst:includeOperator} includes the \textsc{Tavor format} file \enquote{number.tavor}.

\begin{listing}
\caption{Example for include operator}
\label{lst:includeOperator}
\begin{minted}[
  mathescape,
  linenos,
  numbersep=5pt,
  breaklines,
  bgcolor=white,
]{go}
START = ${include "number.tavor"}
\end{minted}
\end{listing}

\paragraph{Path Operator}
\label{subsec:pathOperator}

The \texttt{path operator} traverses a \texttt{list token} based on the described structure. The structure defines the starting value of the traversal, the value which identifies each entry of the \texttt{list token}, how the entries are connected and which values are ending values for the traversal. The \texttt{path operator} has the format \enquote{path from (<starting value>) over (<entry identifier>) connected by (<entry connections>) without(<ending values>)}. All values are \texttt{expressions}. Furthermore, the \texttt{entry connections} and \texttt{ending values} are lists of \texttt{expressions}. The \texttt{entry identifier}, \texttt{entry connections} and \texttt{ending values} have the variable \enquote{e} in their scope which holds the currently traversed entry of the \texttt{list token}.

Listing~\ref{lst:pathOperator} defines a list of connections called \enquote{Pairs}. Each entry in the list \enquote{Pairs} defines the identifier as its first \texttt{token} and the connection as its second \texttt{token}. The used \texttt{path operator} arguments define that all entries are traversed beginning from the value \enquote{2} and ending at the value \enquote{0}. The example holds the string \enquote{103123->231}.

\begin{listing}
\caption{Example for path operator}
\label{lst:pathOperator}
\begin{minted}[
  mathescape,
  linenos,
  numbersep=5pt,
  breaklines,
  bgcolor=white,
]{go}
START = Pairs "->" Path

Path = ${Pairs path from (2) over (e.Item(0)) connect by (e.Item(1)) without (0)}

Pairs = (,
	(1 0),
	(3 1),
	(2 3),
)
\end{minted}
\end{listing}

\paragraph{Not-in Operator}
\label{subsec:notInOperators}

The \texttt{not-in operator} queries the \enquote{Existing} \texttt{token attribute} of a sequence to not include the given list of \texttt{expressions}. A list of \texttt{expressions} begins with the opening parenthesis \enquote{(} and ends with the closing parenthesis \enquote{)}. Each \texttt{expression} is defined without the expression frame \enquote{\$\{...\}}. \texttt{Expressions} are separated by a comma. Listing~\ref{lst:notInOperator} illustrates the \texttt{not-in operator} by issuing two sequence entries and excluding the entry with the value \enquote{2}. The format therefore holds the string \enquote{1, 2 -> 1}.

\begin{listing}
\caption{Example for the not in operator}
\label{lst:notInOperator}
\begin{minted}[
  mathescape,
  linenos,
  numbersep=5pt,
  bgcolor=white,
]{go}
$Id Sequence

START = $Id.Next ", " $Id.Next " -> " ${Id.Existing not in (2)}
\end{minted}
\end{listing}

\section{Variables}
\label{sec:variables}

Every \texttt{token} of a \texttt{token definition} can be saved into a \texttt{variable} which consists of a name and a reference to a \texttt{token usage}. \texttt{Variables} follow the same scope rules as \texttt{token attributes}. It is therefore possible to, for example, define the same \texttt{variable name} more than once in one \texttt{token sequence}. They also do not overwrite \texttt{variable definitions} of parent scopes. \texttt{Variables} can be defined by using the \enquote{<} character after the \texttt{token} that should be saved, then defining the name of the \texttt{variable} and closing with the \enquote{>} character. \texttt{Variables} have a range of \texttt{token attributes} which are listed in Table~\ref{table:variableTokenAttributes}. Listing~\ref{lst:tokenVariable} illustrates the usage of \texttt{variables} by saving the \texttt{String Token} \enquote{text} into the \texttt{variable} \enquote{var}. The \texttt{token} \enquote{Print} uses this \texttt{variable} by embedding the referenced \texttt{token}. The format therefore holds the string \enquote{text->text}. \texttt{Tokens} which are saved to \texttt{variables} are by default relayed to the generation. This means that their usage generates data as usual. Since this is sometimes unwanted, an equal sign \enquote{=} after the \enquote{<} character can be used to omit the relay.

\begin{listing}
\caption{Example for a token variable}
\label{lst:tokenVariable}
\begin{minted}[
  mathescape,
  linenos,
  numbersep=5pt,
  bgcolor=white,
]{go}
START = "text"<var> "->" Print

Print = $var.Value
\end{minted}
\end{listing}

\begin{table}[H]
\caption{Token attributes of variable tokens}
\label{table:variableTokenAttributes}
\center
\begin{tabulary}{\textwidth}{| l | l | J |}
\hline
  \textbf{Attribute}
& \textbf{Arguments}
& \textbf{Description}
\tabularnewline
\hline
  \enquote{Count}
& -
& Holds the count of the referenced token's direct child
\tabularnewline
\hline
  \enquote{Index}
& -
& Holds the index of the referenced token in relation to its parent
\tabularnewline
\hline
  \enquote{Item}
& \enquote{i}
& Holds a child of the referenced token with the index \enquote{i}
\tabularnewline
\hline
  \enquote{Reference}
& -
& Holds a reference to a token which is needed for some operators
\tabularnewline
\hline
  \enquote{Value}
& -
& Embeds a new token based on the referenced token
\tabularnewline
\hline
\end{tabulary}
\end{table}

\section{Statements}
\label{sec:statements}

\texttt{Statements} allow the \textsc{Tavor format} to have a control flow in its \texttt{token definitions} depending on the used tokens and values. All \texttt{statements} start with the opening curly brace \enquote{\{} and end with closing curly brace \enquote{\}}. The \texttt{statement operator} must be defined right after the closing curly brace.

The \texttt{if statement} allows to embed conditions into \texttt{token definitions} and defines an \texttt{if body} which is a scope on its own. The \texttt{token body} lies between an opening \texttt{\{if <condition>\}} \texttt{statement} and an ending \texttt{\{endif\}} \texttt{statement}. The condition can be formed using the \texttt{if statement's operators} which are listed in Table~\ref{table:ifOperators}. Each \texttt{if body} creates a new scope. Listing~\ref{lst:ifStatement} illustrates the \texttt{if statement} by generating the character \enquote{A} only if the variable \enquote{var} is equal to \enquote{1}.

\begin{table}
\caption{Operators for the if statement condition}
\label{table:ifOperators}
\center
\begin{tabulary}{\textwidth}{| l | l | J |}
\hline
  \textbf{Operator}
& \textbf{Usage}
& \textbf{Description}
\tabularnewline
\hline
  \enquote{==}
& \enquote{<token> == <token>}
& Returns true if the value of op1 is equal to op2
\tabularnewline
\hline
  \enquote{defined}
& \enquote{defined <name>}
& Returns true if name is a defined variable
\tabularnewline
\hline
\end{tabulary}
\end{table}

\begin{listing}
\caption{Example for the if statement}
\label{lst:ifStatement}
\begin{gocode}
Choose = 1 | 2 | 3

START = Choose<var> "->" {if var.Value == 1}"A"{endif}
\end{gocode}
\end{listing}

Additional to the \texttt{if statement}, the statements \texttt{else} and \texttt{else if} can be used. They can only be defined inside an \texttt{if body} and always belong to the \texttt{if statement} that the body belongs to. Both statement operators create a new \texttt{if body}.
