\appendix

\chapter{Tavor Framework Pseudo Codes}
\label{chapter:AppendixTavorFramework}

\begin{listing}
\caption{Pseudo Code of AllPermutations fuzzing strategy}
\label{lst:pseudo-code-all-perm-strat}
\begin{gocode}
func fuzz(contin chan struct{}, list []permLevel) {
Step:
	for {
		if len(list[0].children) > 0 {
			fuzz(contin, list[0].children)
		} else {
			reportPermutation(contin)
		}

		list[0].perm++

		if list[0].perm >= list[0].token.Permutations(){
			for i:=1; i < len(list); i++ {
				incremented := incrementChild(list[i].children)
				if incremented {
					resetTokensBeforeIndex(list, i)
					continue Step
				}

				list[i].perm++

				if list[i].perm < list[i].token.Permutations() {
					resetTokensBeforeIndex(list, i)
					setEntry(list, i)
					continue Step
				}
			}
			break Step
		} else {
			setToken(list, 0)
		}
	}
}
\end{gocode}
\end{listing}

\begin{listing}
\caption{Pseudo Code of AllPermutations fuzzing strategy Helper Function}
\label{lst:pseudo-code-all-perm-strat-helper}
\begin{gocode}
func incOne(list []permLevel) bool {
	for {
		if len(list[0].children) > 0 {
			if incOne(list[0].children) {
				return true
			}
		}

		list[0].perm++

		if list[0].perm >= list[0].token.Permutations(){
			for i:=1; i < len(list); i++ {
				if incrementChild(list[i].children) {
					resetTokensBeforeIndex(list, i)

					return true
				}

				list[i].perm++

				if list[i].perm < list[i].token.Permutations() {
					resetTokensBeforeIndex(list, i)
					setEntry(list, i)

					return true
				}
			}
			break Step
		} else {
			setToken(list, 0)

			return true
		}
	}

	return true
}
\end{gocode}
\end{listing}

\chapter{Tavor CLI Command Line Arguments}
\label{chapter:AppendixTavorCLI}

\begin{listing}
\caption{General arguments of the Tavor CLI}
\label{lst:tavor-cli-general-arguments}
\begin{textcode}
General options:
  --debug             Debug log output
  --help              Show this help message
  --verbose           Verbose log output
  --version           Print the version of this program

Global options:
  --seed=             Seed for all the randomness
  --max-repeat=       How many times loops and repetitions should be repeated (default: 2)

Format file options:
  --check             Checks the syntax of the format file and exits
  --format-file=      Input Tavor format file
  --print             Prints the AST of the parsed format file and exits
  --print-internal    Prints the internal AST of the parsed format file and exits
\end{textcode}
\end{listing}

\begin{listing}
\caption{Arguments for \texttt{validate} the command of the Tavor CLI}
\label{lst:tavor-cli-validate-options}
\begin{textcode}
[validate command options]
      --input-file=   Input file which gets parsed and validated via the format file
\end{textcode}
\end{listing}

\begin{listing}
\caption{Example DOT format for the \texttt{graph} command}
\label{lst:tavor-cli-graph-example-dot}
\begin{textcode}
digraph Graphing {
  node [peripheries = 2]; xc4200e7810 xc4200e75b0; node [peripheries = 1];
  node [shape = point] START;
  node [shape = point] xc4200e7800;
  node [shape = point] xc4200e7810;
  node [shape = ellipse];

  xc4200e75b0 [label="f"]
  xc4200e7470 [label="a"]
  xc4200e74b0 [label="b"]
  xc4200e74f0 [label="c"]
  xc4200e7800 [label=""]
  xc4200e7810 [label=""]
  xc4200e7530 [label="d"]
  xc4200e7570 [label="e"]

  START -> xc4200e7470;
  xc4200e7470 -> xc4200e74b0[ style=dotted];
  xc4200e7470 -> xc4200e74f0;
  xc4200e74b0 -> xc4200e74f0[ style=dotted];
  xc4200e7530 -> xc4200e7570;
  xc4200e7800 -> xc4200e7530;
  xc4200e7570 -> xc4200e7810;
  xc4200e7810 -> xc4200e7800[ label="2-4x"];
  xc4200e74f0 -> xc4200e7800;
  xc4200e7810 -> xc4200e75b0[ style=dotted];
}
\end{textcode}
\end{listing}

\begin{landscape}
\begin{listing}
\caption{Arguments for the \texttt{fuzz} command of the Tavor CLI}
\label{lst:tavor-cli-fuzz-options}
\begin{textcode}
[fuzz command options]
      --exec=                                    Execute this binary with possible arguments to test a generation
      --exec-exact-exit-code=                    Same exit code has to be present (default: -1)
      --exec-exact-stderr=                       Same stderr output has to be present
      --exec-exact-stdout=                       Same stdout output has to be present
      --exec-match-stderr=                       Searches through stderr via the given regex. A match has to be present
      --exec-match-stdout=                       Searches through stdout via the given regex. A match has to be present
      --exec-do-not-remove-tmp-files             If set, tmp files are not removed
      --exec-do-not-remove-tmp-files-on-error    If set, tmp files are not removed on error
      --exec-argument-type=                      How the generation is given to the binary (default: stdin)
      --list-exec-argument-types                 List all available exec argument types
      --script=                                  Execute this binary which gets fed with the generation and should return feedback
      --exit-on-error                            Exit if an execution fails
      --filter=                                  Fuzzing filter to apply
      --list-filters                             List all available fuzzing filters
      --strategy=                                The fuzzing strategy (default: random)
      --list-strategies                          List all available fuzzing strategies
      --result-folder=                           Save every fuzzing result with the MD5 checksum as filename in this folder
      --result-extension=                        If result-folder is used this will be the extension of every filename
      --result-separator=                        Separates result outputs of each fuzzing step (default: "\n")

\end{textcode}
\end{listing}
\end{landscape}

\begin{landscape}
\begin{listing}
\caption{Arguments for the \texttt{reduce} command of the Tavor CLI}
\label{lst:tavor-cli-reduce-options}
\begin{textcode}
[reduce command options]
      --exec=                           Execute this binary with possible arguments to test a generation
      --exec-exact-exit-code            Same exit code has to be present
      --exec-exact-stderr               Same stderr output has to be present
      --exec-exact-stdout               Same stdout output has to be present
      --exec-match-stderr=              Searches through stderr via the given regex. A match has to be present
      --exec-match-stdout=              Searches through stdout via the given regex. A match has to be present
      --exec-do-not-remove-tmp-files    If set, tmp files are not removed
      --exec-argument-type=             How the generation is given to the binary (default: stdin)
      --list-exec-argument-types        List all available exec argument types
      --script=                         Execute this binary which gets fed with the generation and should return feedback
      --input-file=                     Input file which gets parsed, validated and delta-debugged via the format file
      --strategy=                       The reducing strategy (default: Linear)
      --list-strategies                 List all available reducing strategies
      --result-separator=               Separates result outputs of each reducing step (default: "\n")

\end{textcode}
\end{listing}
\end{landscape}

\begin{listing}
\caption{Arguments for the \texttt{graph} command of the Tavor CLI}
\label{lst:tavor-cli-graph-arguments}
\begin{textcode}
[graph command options]
      --filter=         Fuzzing filter to apply
      --list-filters    List all available fuzzing filters
\end{textcode}
\end{listing}

\chapter{Case Study: Coin Vending Machine}
\label{chapter:coinVendingMachine}

\begin{listing}
\caption{Semi Automated Delta-Debugging of Coin Vending Machine Case Study}
\label{lst:semi-automated-delta-debug-vending-machine}
\begin{textcode}
credit  0

Do the constraints of the original input still hold for this generation? [yes|no]: no
credit  0
coin    50
credit  50
coin    50
credit  100
vend
credit  0

Do the constraints of the original input still hold for this generation? [yes|no]: no
credit  0
coin    50
credit  50
coin    25
credit  75
coin    25
credit  100
vend
credit  0

Do the constraints of the original input still hold for this generation? [yes|no]: yes
\end{textcode}
\end{listing}


\begin{listing}
\caption{Partial Log of Applying \textsc{go-mutesting} for the Coin Vending Machine Case Study}
\label{lst:mutation-testing-vending-machine}
\begin{textcode}
PASS "/tmp/go-mutesting-291931067//home/zimmski/go/src/coin/implementation.go.0" with checksum 59ff7a970964f4a5bdaaac92d09b8600
PASS "/tmp/go-mutesting-291931067//home/zimmski/go/src/coin/implementation.go.1" with checksum 9bc8a0e74dd28e7377aec58c336d0852
--- /home/zimmski/go/src/coin/implementation.go 2017-11-22 11:33:03.020149906 +0100
+++ /tmp/go-mutesting-291931067//home/zimmski/go/src/coin/implementation.go.2   2017-11-22 11:39:25.736642702 +0100
@@ -39,7 +39,8 @@
        case coin50:
                v.credit += credit
        default:
-               return ErrUnknownCoin
+               _ = ErrUnknownCoin
        }
        return nil
FAIL "/tmp/go-mutesting-291931067//home/zimmski/go/src/coin/implementation.go.2" with checksum 355ad77c0828056fde05b28d923e440c
--- /home/zimmski/go/src/coin/implementation.go 2017-11-22 11:33:03.020149906 +0100
+++ /tmp/go-mutesting-291931067//home/zimmski/go/src/coin/implementation.go.3   2017-11-22 11:39:26.040652417 +0100
@@ -48,7 +48,7 @@
 // Vend executes a vend of the machine if enough credit (100) has been put in and returns true.
 func (v *VendingMachine) Vend() bool {
        if v.credit < 100 {
-               return false
        }
        v.credit -= 100
FAIL "/tmp/go-mutesting-291931067//home/zimmski/go/src/coin/implementation.go.3" with checksum 9bf76c4dd255f2c388659111af37a790
PASS "/tmp/go-mutesting-291931067//home/zimmski/go/src/coin/implementation.go.6" with checksum a42a707de248e2cda066676e24484c6b
\end{textcode}
\end{listing}
